\documentclass[acmsmall, nonacm]{acmart}

\usepackage{graphicx} % Required for inserting images
\usepackage{hyperref}
\usepackage{url}

\begin{document}

\title{Data Management By Machines For Machines}
\author{Michael Cahill}
\affiliation{%
  \institution{\href{mailto:michael.cahill@sydney.edu.au}{michael.cahill@sydney.edu.au},
  University of Sydney}
% \city{Sydney}
  \country{Australia}
}

\date{March 2024}
\maketitle

As data variety, velocity, and volume explode, real-time insights remain critical and the limitations of human-centric data management systems become increasingly evident. In the near future, systems will be constructed and optimized by intelligent machines and specifically tailored for machine-based consumption. This transformation requires a radical rethinking of traditional database interfaces, interactions, and implementation.

Machine-oriented data management systems (MODMS) will leverage artificial intelligence (AI) and machine learning (ML) throughout a system's lifecycle for code generation, correctness and performance testing, autonomous configuration, data structure optimization, and workload management. A profound change lies in the design of MODMS expressly for machine consumption, which allows us to eliminate traditional human-oriented interfaces such as human-readable configuration parameters, query languages, and monitoring.

Recent work has explored the configuration space of key/value stores and demonstrated that finding optimal solutions is feasible when the input workload, budget and performance targets are known in advance. We plan to apply these ideas to the implementation of MODMS, making them easily reconfigurable so that machines can dynamically explore the configuration space and adapt to changes in workloads and budgets. 

Each application using MODMS will have its own preferred data format, which will evolve over time, and MODMS should not impose an arbitrary schema on applications. However, MODMS will store data and process complex queries over structured data on behalf of multiple untrusted tenants, so executing arbitrary code supplied by clients is not safe. Thus, interfaces are still required to specify the computation over  data. But since the consumers of query results are machines, they can perform further computation, opening up new opportunities to optimize w

A fascinating aspect of this paradigm is the potential for machine-generated code to play a significant role in the implementation of MODMS. AI-driven systems could dynamically generate optimized code for tasks that include query execution, indexing, and resource management. This raises intriguing possibilities and important considerations:

\begin{itemize}
    \item Autonomous Configuration and Tuning: Databases self-optimize by dynamically adjusting parameters, data layouts, and indexing, all without human intervention.
    \item Adaptability: MODMS would rapidly adapt to the evolving patterns and usage of the data, with parts of the implementation generated on the fly as needed.
    \item Complexity and Verification: Ensuring the correctness and reliability of dynamically generated code poses new challenges in debugging and verification.
    \item Intelligent Workload Management: AI-powered systems analyze patterns and telemetry, predicting and proactively optimizing resources for anticipated workloads.
\end{itemize}

This shift has far-reaching implications for high-performance transaction systems. By removing human bottlenecks in database management and embracing machine-oriented implementations, we will improve efficiency, adaptability, and autonomy in machine-driven deployments.

This work explores the transformative potential of machines to build and optimize their own database management systems. We investigate the limits of code generation in the DBMS context and address the critical challenges it presents.

\end{document}
